\documentclass{article}
\usepackage[utf8]{inputenc}
\usepackage{amsfonts} % Uso de fontes para conjuntos como R (reais), Z (inteiros), etc.
\usepackage[a4paper, left=20mm, right=20mm, top=20mm, bottom=20mm]{geometry} % Formatação da página

\begin{document}
    \begin{center}
        \section*{INE5416 - Paradigmas da Programação (2015/2)}
        \textbf{\textit{Relatório 7: Módulos} \\
        Caique Rodrigues Marques 13204303}
    \end{center}

O módulo \textit{hyperbolic\_functions.hs} implementado na linguagem de programação Haskell realiza os cálculos das seguintes operações de funções hiperbólicas, sendo $e$ a constante matemática que pode ser calculada pela seguinte série de Taylor $\sum_{n=1}^{\infty}\frac{1}{n!} = \frac{1}{1} + \frac{1}{1.2} + \frac{1}{1.2.3} +...$. \\ \\

\noindent
$Senh(x) = \frac{e^{x} - e^{-x}}{2}$ \\ \\
$Cosh(x) = \frac{e^{x} + e^{-x}}{2}$ \\ \\
$Tanh(x) = \frac{Senh(x)}{Cosh(x)}$ \\ \\
$Coth(x) = \frac{Cosh(x)}{Senh(x)}$ \\ \\
    
No módulo, a determinação da constante $e$ foi estabelecida como $n = 1000$, ou seja, é calculado o somatório da sequência de $1$ até $\frac{1}{1000!}$, devido a esta limitação, a aproximação dos valores tende a ser menor em comparação às implementações nativas do Haskell, que tem até seis casas de aproximação.

\begin{verbatim}
    *HyperbolicFunctions> sinh 1
    1.1752011936438014
    *HyperbolicFunctions> value (Sinh 1)
    1.1752012
\end{verbatim}

As outras funções possíveis de usar são \textit{value(Cosh x)}, \textit{value(Tanh x)} e \textit{value(Coth x)} com $x \in \mathbb{R}$.

\end{document}
