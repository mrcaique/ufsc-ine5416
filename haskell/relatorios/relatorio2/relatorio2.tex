\documentclass{article}
\usepackage[utf8]{inputenc}
\usepackage{indentfirst} %Indentar primeiro parágrafo (desativado por padrão)
\usepackage{listings}
\usepackage[a4paper, left=20mm, right=20mm, top=20mm, bottom=20mm]{geometry}

\begin{document}
\lstset{language=C}
\begin{center}
    \section*{INE5416 - Paradigmas da Programação (2015/2)}
    \textbf{\textit{Relatório 2: Estrutura das Linguagens} \\
    Caique Rodrigues Marques 13204303}
\end{center}
    Neste roteiro de prática, o aluno deve pesquisar sobre estrutura das
    linguagens de programação, principalmente como os arquivos de gramática são
    feitos e gerados.
\section*{Parte 1}
    A linguagem tem como principal função comunicar e expressar, assim, uma
    linguagem de programação serve para comunicar à máquina quais operações a
    fazer.

    No entanto, máquinas reconhecem apenas o binário, a sua língua materna, por
    isso os compiladores traduzem as linguagens de programação de alto nível,
    mais amigável à pessoa, para o binário. Entre as duas camadas, linguagens
    de alto nível e linguagem binária, tem a linguagem de baixo-nível, o
    Assembly, que é pouco amigável e pode variar de máquina para máquina.

    Para comunicar à máquina é necessário que a linguagem de programação
    usa\-da esteja bem escrita, seguindo a sintaxe e contenha o conjunto léxico
    definido. Softwares são utilizados para garantir as qualidades desejadas:
\begin{itemize}
    \item \textit{\textbf{Flex}} (\textit{Fast Lexical Analyzer}): Uma
        ferramenta para gerar scanners. Um scanner, às vezes chamado como
        "tokenizador", é um programa que reconhece padrões léxicos no
        código-fonte, ou seja, verifica se o vocabulário está de acordo com os
        padrões definidos pela linguagem. O programa lê um dado arquivo de
        entrada ou seu próprio padrão de entrada, se não houver arquivos
        direcionados a ele, para criar descrições para gerar um scanner. As
        descrições são em formatos de pares de expressões regulares e código C,
        chamados de regras. Flex gera uma saída em código-fonte C, lex.yy.c por
        padrão, que define uma função yylex(). Este arquivo pode ser compilado
        e "linkado" com a biblioteca flex runtime para gerar um arquivo
        executável. Flex foi escrito em 1987 por Vern Paxson, na linguagem C, o
        software surgiu como uma alternativa open-source ao lex.

    \item \textit{\textbf{Lex}}: Gera programas para ser usado como analisador
        léxico de textos. Os arquivos de entrada possuem expressões regulares
        para serem encontradas e ações escritas em C mostrando o que executar
        quando uma expressão for encontrada. Um programa em C é gerado, chamado
        de lex.yy. Quando executado, o programa copia partes não reconhecidas
        da entrada para a saída, e executa a ação em código C para cada
        expresão regular reco\-nhecida. Lex foi escrito por Mike Lesk e Eric
        Schmidt em 1975.
    
    \item \textit{\textbf{Yacc}} (\textit{Yet Another Compiler Compiler}):
        Converte uma gramática livre-de-contexto e uma tradução de código para
        um conjunto de tabelas para um analisador sintático LR (left-right) e
        tradutor. A gramática pode ser ambígua, as regras especificadas
        previamente são usadas para quebrar as ambiguidades. O arquivo gerado,
        y.tab.c, pode ser compilado por um compilador C para produzir um
        programa chamado yyparse. Este programa pode ser carregado com um
        analisador léxico (lex ou o flex). Yacc foi desenvolvido no começo dos
        anos 1970 por Stephen C. Johnson da AT\&T Corp. e escrito em B, depois
        reescrito em C.

    \item \textit{\textbf{GNU Bison}}: É um gerador de analisador sintático
        (parser) no estilo do Yacc. Ele é compatível com arquivos designados ao
        Yacc, portanto, os arquivos possuem a extensão .y. No entanto, o
        arquivo gerado não possui nomes fixos (como yyparse no Yacc), usa
        prefixos do arquivo de entrada. Bison foi escrito por Robert Corbett em
        1988 e ganhou a compatibilidade com o Yacc graças a Richard Stallman.
\end{itemize}

\newpage
\section*{Parte 2}
\subsection*{y.tab.h}
    O arquivo é gerado pelo analisador sintático Yacc/Bison, ele contém as
    definições de cada tipo correspondendo a um valor decimal (o primeiro valor
    é 257, número após o último elemento da tabela ASCII).
\subsection*{scan.l}
    Os arquivos de extensão .l são arquivos Lex/Flex, como difinido antes, eles
    são usados para definir a estrutura léxica do programa, neste caso, da
    própria linguagem C. Em scan.l, como qualquer arquivo lex, é divido em três
    seções, separadas por "\%\%", onde, de cima para baixo, começa na seção de
    definição, depois, seção de regras e, por fim, seção de código em C.
\begin{verbatim}
    D           [0-9]
    L           [a-zA-Z_]
    H           [a-fA-F0-9]
    E           [Ee][+-]?{D}+
    FS          (f|F|l|L)
    IS          (u|U|l|L)*

    %{
    #include <stdio.h>
    #include "y.tab.h"

    void count();
    %} \end{verbatim}
    
    Na seção de definição (acima), são definidos o conjunto de possíveis
    símbolos para uma expressão regular, ou seja, o alfabeto. Nas últimas cinco
    linhas há uma seção de código em C, onde há duas inclusões, o padrão
    standard do C e o arquivo y.tab.h. Por fim, a função count é chamada, esta
    foi definida na última seção do arquivo, que será explicada adiante.
    
    Na seção de regras, apenas relacionado os caracteres (o alfabeto) com os
    tipos definidos no arquivo y.tab.c. Antes de retornar com o inteiro
    correspondente ao caractere, a função count() é chamada.
    
    Na seção de código C é onde mostra como o programa deve se comportar. Há
    quatro funções definidas: yywrap(), comment(), count() e check\_type().
\begin{itemize}
\begin{lstlisting}[frame=single]
yywrap() {
    return(1);
}
    
comment() {
    char c, c1;
    
loop:
    while ((c = input()) != '*' && c != 0)
        putchar(c);

    if ((c1 = input()) != '/' && c != 0) {
        unput(c1);
        goto loop;
    }

    if (c != 0)
        putchar(c1);
}
    
int column = 0;

void count() {
    int i;

    for (i = 0; yytext[i] != '\0'; i++)
        if (yytext[i] == '\n')
            column = 0;
        else if (yytext[i] == '\t')
            column += 8 - (column % 8);
        else
            column++;

    ECHO;
}
\end{lstlisting}
    \item A função yywrap() é a versão default onde apenas retorna o inteiro 1.
        Ela é usada após o scanner do lex chegar ao fim do arquivo;
    \item A função comment(), é definido o sistema de comentários que são
        marcados como os símbolos que estiverem entre "/*" e "*/",
        representando o início e o fim do comentário, respectivamente;
    \item A função count() é o analisador sintático, ele verifica a palavra
        recebida, num loop, até o final "\textbackslash0". A varíavel "column"
        representa onde o cursor está posicionado e, caso o texto lido seja um
        "\textbackslash n", o cursor volta ao início, ou seja, foi acrescentada
        um nova linha. Se o texto lido foi um "\textbackslash t", significando
        "tab" é adicionado um espaçamento a mais na posição do cursor;
    \item A função check\_type() apenas retorna o tipo (ou o identificador) do
        ca\-ractere digitado (seja int, enum, char, etc.).
\end{itemize}
    
\subsection*{gram.y}
    O arquivo Bison/Yacc gerado é o analisador sintático dos arquivos gerados
    pelo lex, na primeira seção é estabelecido os tokens que a linguagem deve
    conter, ou seja, os tipos presentes. Na segunda parte, após o "\%\%", é
    estabelecido a sintaxe das expressões usadas na linguagem, ou seja, a
    "forma" da expressão. Exemplos:
\begin{verbatim}
    and_expr
        : equality_expr
        | and_expr '&' equality_expr
        ;
\end{verbatim}
    Acima mostra como deve ser uma expressão and ("e"), com duas proposições a
    serem comparadas entre o sinal de "\&".
\begin{verbatim}
    init_declarator
        : declarator
        | declarator '=' initializer
        ;
\end{verbatim}
    É definido a declaração de uma varíavel: definindo o nome da variável e
    após o sinal "=" é o inicializador.
    
\section*{Fontes}
\begin{itemize}
    \item http://flex.sourceforge.net/;
    \item manual page do Bison (http://dinosaur.compilertools.net/bison/manpage.html);
    \item manual page do Lex (http://plan9.bell-labs.com/magic/man2html/1/lex);
    \item manual page do Yacc (http://plan9.bell-labs.com/magic/man2html/1/yacc).
\end{itemize}
\end{document}
