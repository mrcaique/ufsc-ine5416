\documentclass{article}
\usepackage[utf8]{inputenc}
\usepackage[a4paper, left=20mm, right=20mm, top=20mm, bottom=20mm]{geometry} % Formatação da página

\begin{document}
    \begin{center}
        \section*{INE5416 - Paradigmas da Programação (2015/2)}
        \textbf{\textit{Relatório 10: Mônadas} \\
        Caique Rodrigues Marques 13204303}
    \end{center}
    
    \section*{Parte 1}
        \begin{itemize}
            \item Homologia algébrica envolve definições abstratas utilizadas para o estudo desta área. Um
            espaço topológico pode ser definido como um conjunto de pontos e seus vizinhos, que todos
            satisfazem uma série de axiomas; permitem formalizar conceitos de continuidade e convergência. A
            homologia algébrica é a área de matemática que estuda estes espaços em uma visão algébrica.

            \item Functor, em teoria de categorias (área da matemática responsável em estudar as estruturas
            abstratas, os morfismos), é um tipo de mapeamento entre duas categorias, preservando a estrutura.
            Functor pode ser considerado como um homomorfismo de categorias.

            \item Mônadas, em programação funcional, são estruturas que permitem a realização de tarefas numa
            sequência de passos, como programação imperativa, essas estruturas permitem ao programador
            adicionar tarefas quando achar necessário. Um functor funciona como um mapeador (\texttt{map}) de
            quaisquer tipos de variáveis passados como parâmetro.
        \end{itemize}
        
    \section*{Parte 2}
        Facilmente percebe-se o uso de mônadas no código em Haskell, outro ponto a notar é a semelhança ao
        paradigma imperativo: atribuições em sequência de condições, onde nesta não faz nada se a divisão for
        por zero ou realiza apenas a divisão em si. A seguir, uma simples definição de soma de dois
        elementos. Na função \texttt{resist} é onde ocorre o cálculo da resistência, nota-se a vantagem é que
        o programador não precisa se preocupar com o caso da divisão por zero, pois o próprio programa já
        trata disso conforme especificado na função \texttt{//}.
\end{document}
